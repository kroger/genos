%% Local IspellDict: brasileiro

\chapter{Introdução ao estudo de ...}
\label{chap:introducao}


\section{Objetivos}
\label{sec:objetivos}

Os principais objetivos deste trabalho foram:

\begin{enumerate}
\item Foo
\item Bar
\end{enumerate}

Este trabalho teve como objetivos secundários:

\begin{enumerate}
\item Bla
\item Ga
\end{enumerate}

\section{Justificativa}
\label{sec:justificativa}

Este trabalho é importante porque...

\section{Procedimentos metodológicos}
\label{sec:metodologia}

Para atingir os objetivos mencionados na seção~\ref{sec:objetivos} foi
necessário...

\begin{enumerate}
\item Foo
\item Bar
\end{enumerate}

\section{Estrutura dos capítulos}
\label{sec:estrutura-capitulos}

Este trabalho contém os seguintes capítulos:

\begin{enumerate}
\item Introdução. Visão geral de...
\item Revisão de literatura. ...
\end{enumerate}

\chapter{Revisão de Literatura}
\label{chap:revisao-literatura}

Os principais conceitos ...

Lorem ipsum dolor sit amet, consectetur adipiscing elit. Sed pretium
sem vel massa fringilla vehicula. Fusce posuere cursus metus ac
pellentesque. Quisque varius ultrices nulla. Mauris aliquam viverra
nisi, tristique pharetra nibh ornare id. In dignissim, leo scelerisque
ornare imperdiet, leo est euismod felis, ut mollis nibh justo a elit.
Morbi scelerisque turpis diam, nec volutpat lorem pulvinar sit amet.
Sed facilisis, tortor quis malesuada ultricies, dui lacus mollis
tortor, non malesuada odio lectus quis nisl. Nulla faucibus quis risus
iaculis eleifend. Aenean mattis sapien id leo sollicitudin, tempus
tempus lacus elementum. Proin cursus felis et metus viverra volutpat.
\cite{Sampaio2012b}.


\chapter{Discussão}
\label{chap:discussao}

No capítulo~\ref{chap:introducao} eu levantei as hipóteses...


\chapter{Aplicações práticas da Teoria...}
\label{chap:aplicacao}


\chapter{Conclusões}
\label{chap:conclusoes}

Este trabalho consistiu ...

\appendix
\chapter{Partituras}
\label{chap:partituras}

%% Usar para incluir partituras
% \includegraphics[scale=.60,page=1]{arquivo-pdf-da-partitura-1}\pagebreak\par
% \score{arquivo-pdf-da-partitura-1}{2,3,4}
% \score{arquivo-pdf-da-partitura-2}{2,3,4}

\chapter{Documentos...}
\label{cha:documentos}

\singlespacing

\bibliographystyle{kchicago}
\bibliography{referencias}
