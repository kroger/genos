\documentclass{ppgmus}
\usepackage[T1]{fontenc}
\usepackage{setspace}
\usepackage{colortbl}
% \usepackage{parskip}
\usepackage{listings}
\usepackage{kchicago}
\usepackage{times}
\usepackage[pdfpagelabels=true]{hyperref}
\usepackage[brazil]{translator}

% conta subsubseções
\setcounter{secnumdepth}{3}

\usepackage[scaled=.75]{beramono}
\usepackage{ltablex}
\usepackage{rotating}
\usepackage[nonumberlist]{glossaries}
\usepackage{ulem}
\usepackage{multirow}
\usepackage{pgffor}
\usepackage[portuguese,algochapter]{algorithm2e}
\usepackage{amsfonts}
\usepackage{amsmath}

%% Termos estrangeiros
\newcommand{\eng}[1]{\textit{#1}}
%% Nomes de obras
\newcommand{\opus}[1]{\textit{#1}}
%% Dinâmicas
\newcommand{\dyn}[1]{\textbf{\textit{#1}}}

% usar para citação integral indentada com tradução
\newcommand{\citacaolonga}[3]{
  \singlespacing
  \begin{quote}
    \normalsize
    \selectlanguage{brazil}
    {#2}\footnote{
      \selectlanguage{english}
      ``{#1}''.
    } \selectlanguage{brazil} #3.
  \end{quote}
  \doublespacing
}

\newcommand{\citacaolongabr}[2]{
  \singlespacing
  \begin{quote}
    \normalsize
    \selectlanguage{brazil}
    {#1} #2.
  \end{quote}
  \doublespacing
}

\newcommand{\citacaoinline}[3]{
  ``#2''\footnote{
    \selectlanguage{english}
    ``{#1}''.} \selectlanguage{brazil} #3.
}

\newcommand{\traducaolonga}[2]{
  \singlespacing
  \begin{quote}
    \normalsize
   \selectlanguage{brazil}
    {#2}\footnote{
      \selectlanguage{english}
      ``{#1}''.
    }.
    \selectlanguage{brazil}.
  \end{quote}
  \doublespacing
}

\newcommand{\traducaoinline}[2]{
  ``#2''\footnote{
    \selectlanguage{english}
    ``{#1}''.}
  \selectlanguage{brazil}
}

%% Usar para inserir arquivo pdf com múltiplas páginas
\newcommand{\score}[2]{
  \foreach \i in {#2}{
    \includegraphics[scale=.80,page=\i]{#1}\pagebreak\par
  }
}

%% cria glossários
\makeglossaries

%% Inserir separação de sílaba não prevista
\hyphenation{car-di-na-li-da-de}

%% Local IspellDict: brasileiro

\newglossaryentry{termo}{
        name={termo como aparece},
        description={descrição}
}


\titulo{Título da Tese}
\autor{Autor da tese}
\ano{2013}
\tipo{doutorado}
\area{Composição}
\lugar{Salvador}
\orientador{Prof. Dr. Orientador}
\mes{Fevereiro}

\textoDedicatoria{A fulano, ciclano e beltrano.}

\textoEpigrafe{Bla bla bla bla bla bla bla bla bla bla bla bla bla bla
  bla bla bla bla bla bla bla bla bla bla bla bla bla bla bla bla bla
  bla bla bla bla bla bla bla bla bla bla bla bla bla bla bla bla bla
  bla bla bla.}

\autorEpigrafe{Confucio}

\dataDefesa{Salvador}
\localDefesa{28 de fevereiro de 2013}

% banca
\bancaiNome{Pedro Ribeiro Kroger Júnior}
\bancaiTitulo{Doutor em Composição}
\bancaiEstudo{Universidade Federal da Bahia}
\bancaiAtuacao{Universidade Federal da Bahia}

\bancaiiNome{Liduino José Pitombeira de Oliveira}
\bancaiiTitulo{Doutor em Composição}
\bancaiiEstudo{\textit{Louisianna State University}, EUA}
\bancaiiAtuacao{Universidade Federal de Campina Grande}

\bancaiiiNome{Paulo Costa Lima}
\bancaiiiTitulo{Doutor em Educação}
\bancaiiiEstudo{Universidade Federal da Bahia}
\bancaiiiAtuacao{Universidade Federal da Bahia}

\bancaivNome{Lucas Robatto}
\bancaivTitulo{Doutor em Flauta}
\bancaivEstudo{\textit{University of Washington}, EUA}
\bancaivAtuacao{Universidade Federal da Bahia}

\bancavNome{Hugo Leonardo Ribeiro}
\bancavTitulo{Doutor em Etnomusicologia}
\bancavEstudo{Universidade Federal da Bahia}
\bancavAtuacao{Universidade de Brasília}

% ficha catalográfica. Comente cutter para remover ficha
\cutter{Cutter}
\cdd{cdd}
\cdu{cdu}
\chavei{Composição (Música)}
\chaveii{Teoria musical}
\chaveiii{Contornos Musicais}
\chaveiv{Título}

\begin{document}

\frontmatter

\maketitle

\chapter*{Agradecimentos}
\label{cha:agradecimentos}
%% Local IspellDict: brasileiro

Agradeço a...

\chapter*{Resumo}
\label{cha:resumo}
%% Local IspellDict: brasileiro

Lorem ipsum dolor sit amet, consectetur adipiscing elit. Sed pretium
sem vel massa fringilla vehicula. Fusce posuere cursus metus ac
pellentesque. Quisque varius ultrices nulla. Mauris aliquam viverra
nisi, tristique pharetra nibh ornare id. In dignissim, leo scelerisque
ornare imperdiet, leo est euismod felis, ut mollis nibh justo a elit.
Morbi scelerisque turpis diam, nec volutpat lorem pulvinar sit amet.
Sed facilisis, tortor quis malesuada ultricies, dui lacus mollis
tortor, non malesuada odio lectus quis nisl. Nulla faucibus quis risus
iaculis eleifend. Aenean mattis sapien id leo sollicitudin, tempus
tempus lacus elementum. Proin cursus felis et metus viverra volutpat.

Mauris ipsum erat, mollis ultrices nunc in, blandit condimentum magna.
Etiam iaculis dui eget turpis malesuada, nec adipiscing magna
facilisis. Sed dictum vel orci eget rhoncus. Suspendisse vel libero
velit. Morbi ut dictum massa. Nam vel nibh vel quam tristique
consectetur. Nulla rhoncus magna sed varius malesuada. Mauris tempus
leo ut lectus pellentesque blandit. Sed porttitor suscipit metus ut
sagittis. Mauris convallis euismod mattis. Nunc vitae velit at lectus
sollicitudin faucibus vel egestas dolor. Nulla pulvinar arcu at
interdum dictum. Suspendisse a dapibus sapien, at gravida elit. Ut eu
cursus velit. Cras ac tellus pretium, eleifend urna id, lacinia
tortor. Ut blandit mi non elit luctus, eu ultricies nibh mattis.

Palavras-chave: Ipsum Lorem

\chapter*{Abstract}
\label{cha:abstract}
%% Local IspellDict: english

Lorem ipsum dolor sit amet, consectetur adipiscing elit. Sed pretium
sem vel massa fringilla vehicula. Fusce posuere cursus metus ac
pellentesque. Quisque varius ultrices nulla. Mauris aliquam viverra
nisi, tristique pharetra nibh ornare id. In dignissim, leo scelerisque
ornare imperdiet, leo est euismod felis, ut mollis nibh justo a elit.
Morbi scelerisque turpis diam, nec volutpat lorem pulvinar sit amet.
Sed facilisis, tortor quis malesuada ultricies, dui lacus mollis
tortor, non malesuada odio lectus quis nisl. Nulla faucibus quis risus
iaculis eleifend. Aenean mattis sapien id leo sollicitudin, tempus
tempus lacus elementum. Proin cursus felis et metus viverra volutpat.

Mauris ipsum erat, mollis ultrices nunc in, blandit condimentum magna.
Etiam iaculis dui eget turpis malesuada, nec adipiscing magna
facilisis. Sed dictum vel orci eget rhoncus. Suspendisse vel libero
velit. Morbi ut dictum massa. Nam vel nibh vel quam tristique
consectetur. Nulla rhoncus magna sed varius malesuada. Mauris tempus
leo ut lectus pellentesque blandit. Sed porttitor suscipit metus ut
sagittis. Mauris convallis euismod mattis. Nunc vitae velit at lectus
sollicitudin faucibus vel egestas dolor. Nulla pulvinar arcu at
interdum dictum. Suspendisse a dapibus sapien, at gravida elit. Ut eu
cursus velit. Cras ac tellus pretium, eleifend urna id, lacinia
tortor. Ut blandit mi non elit luctus, eu ultricies nibh mattis.

Keywords: Ipsum lorem

% conta subsubseções
\setcounter{tocdepth}{3}

%% remove sumário do sumário
\addtocontents{toc}{\protect\setcounter{tocdepth}{-2}}
\tableofcontents
\addtocontents{toc}{\protect\setcounter{tocdepth}{5}}

\listoftables
\listoffigures
\listofalgorithms

%% insere glossário
\printglossaries

%% É importante ter esse mainmatter para indicar que começa o corpo do
%% texto
\mainmatter

\section{Introdução ao tema}
\label{sec:introducao}

Lorem ipsum dolor sit amet, consectetur adipiscing elit. Nunc iaculis
eleifend commodo. Nulla facilisi. Cum sociis natoque penatibus et
magnis dis parturient montes, nascetur ridiculus mus. Sed mauris orci,
dapibus eget commodo laoreet, pellentesque et diam. Pellentesque est
elit, molestie eget dolor ac, malesuada mollis turpis. Aenean ornare
viverra consectetur. Cras est velit, vulputate et odio vel, feugiat
tempus lorem. Quisque tortor leo, sollicitudin non pretium id,
fringilla vitae ligula. Maecenas nunc lectus, scelerisque id convallis
ac, euismod et tortor. Quisque sed pulvinar lectus. Sed at magna
viverra, fermentum sapien malesuada, tincidunt turpis. Quisque sit
amet facilisis augue. Nullam quis dignissim diam.

%% Problema a ser resolvido

Ut metus felis, bibendum vel lectus a, aliquam sagittis elit. Cras
rhoncus mollis justo eget mattis. Vestibulum fringilla tellus vitae
euismod faucibus. Pellentesque at dignissim leo. Aliquam in facilisis
quam. Phasellus in ante placerat, hendrerit metus ac, ultricies augue.
Nunc auctor quis nisi eu consectetur.

\section{Objetivos}
\label{sec:objetivos}

%% Objetivo principal
O principal objetivo deste projeto é....

Ut metus felis, bibendum vel lectus a, aliquam sagittis elit. Cras
rhoncus mollis justo eget mattis. Vestibulum fringilla tellus vitae
euismod faucibus. Pellentesque at dignissim leo. Aliquam in facilisis
quam. Phasellus in ante placerat, hendrerit metus ac, ultricies augue.
Nunc auctor quis nisi eu consectetur.

%% Objetivos secundários
São objetivos secundários deste projeto:

\begin{enumerate}
\item Objetivo secundário
\end{enumerate}

\section{Justificativa}
\label{sec:justificativa}

Este projeto é importante porque...

Maecenas vel dignissim lorem, non varius libero. Morbi mauris diam,
pretium hendrerit varius ac, malesuada placerat arcu. Proin sit amet
nisi felis. Nullam id lobortis mauris. In enim felis, elementum sit
amet augue non, ullamcorper consequat purus. Maecenas egestas dolor
lectus, vel vestibulum leo rutrum non. Sed varius suscipit ante, nec
lobortis nisi dapibus id. Pellentesque quis imperdiet erat, ac
consequat tortor. Sed sem velit, tristique quis tempor a, interdum sed
magna. Vivamus vitae purus vitae risus molestie ultrices.

\section{Fundamentação teórica}
\label{sec:fundamentacao}

Maecenas vel dignissim lorem, non varius libero. Morbi mauris diam,
pretium hendrerit varius ac, malesuada placerat arcu. Proin sit amet
nisi felis. Nullam id lobortis mauris. In enim felis, elementum sit
amet augue non, ullamcorper consequat purus. Maecenas egestas dolor
lectus, vel vestibulum leo rutrum non. Sed varius suscipit ante, nec
lobortis nisi dapibus id. Pellentesque quis imperdiet erat, ac
consequat tortor. Sed sem velit, tristique quis tempor a, interdum sed
magna. Vivamus vitae purus vitae risus molestie ultrices.

Integer accumsan diam justo, at tincidunt felis pellentesque a. Nulla
nibh felis, malesuada vel convallis a, interdum vel arcu. In lobortis
ante eu consectetur pellentesque. Aliquam erat volutpat. Vivamus erat
diam, volutpat et eros vel, vehicula luctus sapien. Cras aliquam
dapibus tristique. Morbi felis urna, vulputate nec odio in, porta
gravida turpis. Donec tristique diam leo, sed hendrerit lorem tempus
vel. Proin vestibulum suscipit leo id gravida. Nunc ullamcorper luctus
faucibus. Integer in eleifend lorem, vitae tempus sem. Nulla facilisi.
Ut venenatis mi et purus lobortis tempor consectetur eu enim. Aliquam
eros dui, fringilla placerat lorem ac, ultrices vulputate dolor. Duis
venenatis eros ut rutrum viverra.

\citacaolonga{Maecenas vel dignissim lorem, non varius libero. Morbi mauris diam,
pretium hendrerit varius ac, malesuada placerat arcu. Proin sit amet
nisi felis. Nullam id lobortis mauris. In enim felis, elementum sit
amet augue non, ullamcorper consequat purus. Maecenas egestas dolor
lectus, vel vestibulum leo rutrum non. Sed varius suscipit ante, nec
lobortis nisi dapibus id. Pellentesque quis imperdiet erat, ac
consequat tortor. Sed sem velit, tristique quis tempor a, interdum sed
magna. Vivamus vitae purus vitae risus molestie ultrices}
{Maecenas vel dignissim lorem, non varius libero. Morbi mauris diam,
pretium hendrerit varius ac, malesuada placerat arcu. Proin sit amet
nisi felis. Nullam id lobortis mauris. In enim felis, elementum sit
amet augue non, ullamcorper consequat purus. Maecenas egestas dolor
lectus, vel vestibulum leo rutrum non. Sed varius suscipit ante, nec
lobortis nisi dapibus id. Pellentesque quis imperdiet erat, ac
consequat tortor. Sed sem velit, tristique quis tempor a, interdum sed
magna. Vivamus vitae purus vitae risus molestie ultrices}
{\cite{sampaio12:teoria}}

Integer accumsan diam justo, at tincidunt felis pellentesque a. Nulla
nibh felis, malesuada vel convallis a, interdum vel arcu. In lobortis
ante eu consectetur pellentesque. Aliquam erat volutpat. Vivamus erat
diam, volutpat et eros vel, vehicula luctus sapien. Cras aliquam
dapibus tristique. Morbi felis urna, vulputate nec odio in, porta
gravida turpis. Donec tristique diam leo, sed hendrerit lorem tempus
vel. Proin vestibulum suscipit leo id gravida. Nunc ullamcorper luctus
faucibus. Integer in eleifend lorem, vitae tempus sem. Nulla facilisi.
Ut venenatis mi et purus lobortis tempor consectetur eu enim. Aliquam
eros dui, fringilla placerat lorem ac, ultrices vulputate dolor. Duis
venenatis eros ut rutrum viverra.

\section{Metodologia}
\label{sec:metodologia}

Para alcançar os objetivos deste projeto...

In porta purus urna, sed congue magna dictum ut. Nullam bibendum
fermentum velit, quis eleifend purus luctus in. Morbi luctus lorem
nisl, vel scelerisque nisi elementum id. Cras id urna mattis,
hendrerit turpis sagittis, malesuada orci. Quisque dictum metus a
libero dictum dapibus. Maecenas lacinia porttitor nulla at pharetra.
Nam et porta dolor, eget laoreet sem.

\section{Resultados esperados}
\label{sec:resultados-esperados}

Esperamos com este projeto...

Integer accumsan diam justo, at tincidunt felis pellentesque a. Nulla
nibh felis, malesuada vel convallis a, interdum vel arcu. In lobortis
ante eu consectetur pellentesque. Aliquam erat volutpat. Vivamus erat
diam, volutpat et eros vel, vehicula luctus sapien. Cras aliquam
dapibus tristique. Morbi felis urna, vulputate nec odio in, porta
gravida turpis. Donec tristique diam leo, sed hendrerit lorem tempus
vel. Proin vestibulum suscipit leo id gravida. Nunc ullamcorper luctus
faucibus. Integer in eleifend lorem, vitae tempus sem. Nulla facilisi.
Ut venenatis mi et purus lobortis tempor consectetur eu enim. Aliquam
eros dui, fringilla placerat lorem ac, ultrices vulputate dolor. Duis
venenatis eros ut rutrum viverra.

\section{Estrutura de tópicos da tese}
\label{sec:estrutura-de-topicos}

A tese resultante deste projeto terá a seguinte estrutura de tópicos

\begin{enumerate}
\item Introdução, objetivos e justificativa
\item Fundamentação teórica
\end{enumerate}


\section{Cronograma}
\label{sec:cronograma}

O cronograma deste projeto está organizado em semestres e inclui as seguintes tarefas:

\begin{enumerate}
\item Créditos em disciplinas do Programa de Pós-Graduação em Música
\item Análise...
\item Qualificativo
\item Redação do texto da tese
\end{enumerate}



\begin{center}
  \begin{tabular}{r|cccccccc}
    \multirow{2}{*}{Atividade} & \multicolumn{8}{c}{Semestre} \\
    \hhline{~--------}
    &1&2&3&4&5&6&7&8\\
    \hline
    Créditos em disciplins&\ok&\ok&\ok&\ok&&&&\\
    Revisão de literatura&\ok&\ok&\ok&&&&&\\
    Qualificativo&&&&&&\ok&&\\
  
  \end{tabular}
\end{center}

%% Inserir itens da bibliografia não citados no texto
\nocite{*}

%% Desnecessário inserir isso. Serve para o Emacs:
%% (setq reftex-default-bibliography '("references.bib"))



\end{document}
