\usepackage[scaled=.75]{beramono}
\usepackage{ltablex}
\usepackage{rotating}
\usepackage[nonumberlist]{glossaries}
\usepackage{ulem}
\usepackage{multirow}
\usepackage{pgffor}
\usepackage[portuguese,algochapter]{algorithm2e}
\usepackage{amsfonts}
\usepackage{amsmath}

%% Termos estrangeiros
\newcommand{\eng}[1]{\textit{#1}}
%% Nomes de obras
\newcommand{\opus}[1]{\textit{#1}}
%% Dinâmicas
\newcommand{\dyn}[1]{\textbf{\textit{#1}}}

% usar para citação integral indentada com tradução
\newcommand{\citacaolonga}[3]{
  \singlespacing
  \begin{quote}
    \normalsize
    \selectlanguage{brazil}
    {#2}\footnote{
      \selectlanguage{english}
      ``{#1}''.
    } \selectlanguage{brazil} #3.
  \end{quote}
  \doublespacing
}

\newcommand{\citacaolongabr}[2]{
  \singlespacing
  \begin{quote}
    \normalsize
    \selectlanguage{brazil}
    {#1} #2.
  \end{quote}
  \doublespacing
}

\newcommand{\citacaoinline}[3]{
  ``#2''\footnote{
    \selectlanguage{english}
    ``{#1}''.} \selectlanguage{brazil} #3.
}

\newcommand{\traducaolonga}[2]{
  \singlespacing
  \begin{quote}
    \normalsize
   \selectlanguage{brazil}
    {#2}\footnote{
      \selectlanguage{english}
      ``{#1}''.
    }.
    \selectlanguage{brazil}.
  \end{quote}
  \doublespacing
}

\newcommand{\traducaoinline}[2]{
  ``#2''\footnote{
    \selectlanguage{english}
    ``{#1}''.}
  \selectlanguage{brazil}
}

%% Usar para inserir arquivo pdf com múltiplas páginas
\newcommand{\score}[2]{
  \foreach \i in {#2}{
    \includegraphics[scale=.80,page=\i]{#1}\pagebreak\par
  }
}

%% cria glossários
\makeglossaries

%% Inserir separação de sílaba não prevista
\hyphenation{car-di-na-li-da-de}
